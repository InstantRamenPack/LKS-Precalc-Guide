\chapter{Tricky Trigonometry}
- explain that this mostly coveres derivatives
\section{Radians}
- explain what a radian is
- estimating radians
- problem
\section{Intro to Trigonometric Functions}
- introductoin to trigonometry Functions (what they are, how to get)
\subsection{Properties of Trigonometric Functions}
- Amplitude
- Vertical shift
- Frequency/Period
- Phase 
\subsection{Estimation}
- how to estimate trig values

\subsection{Trigonometric Equations}
- solving them
- problem
\section{Trigonometric Identities}
- derive sin2x+cos2x=1
- derive sinpi/2-x=cosx
\section{The Unit Circle}
- unit circle what it is
- trig functions on a unit cricle
- same value diff parity
- laws on circle
- problem
\subsection{Special Values of Trigonometric Functions}
- pi, 0, pi/6, pi/3, etc
- how they appear on a unit circle
- problem
\section{Trigonometric Functions and their Derivatives}
\subsection{sin(x)}
- derive derivative of sin(x)
\subsection{cos(x)}
- derive derivative of cos(x)
\subsection{tan(x)}
- derive derivative of tan(x)
\section{More Trigonometric Functions}
- sec, csc, cot
- their derivative 
- problem
\section{Inverse Trigonometric Functions}
- arcsin arccos arctan
- their Derivative
- problem

\begin{subappendices}
    \section{Angle Addition}
    - derive angle Addition
    - problem
\end{subappendices}