\chapter{Sequences and Series}
What is a sequence? What is a series? To put it simply, a sequence is an ordered set of numbers. A series is the sum of an ordered set of numbers. These can be finite, or infinite. As a rule of thumb, if a sequence ends in dots, it is infinite. Sequences can be a random set of numbers, but for this chapter, we will be working with real numbers.
\begin{example}[Sequence of Positive Integers]
    The sequence of the first $10$ positive integers is
    \[a_n=\{1,2,\dots,9,10\}.\]
    Here, $a_1$ denotes the first element of the sequence; in this example, $a_1=1$. We will use this defintion for the rest of this chapter.
\end{example}
I know all you coders may be writhing in pain, but we will stick to that definition of $a_n$ for now. Now that we know what a sequence is, we need to define a partial sum.
\begin{definition}[Partial Sums]
    For a sequence $a_n$, the $n$th partial sum is
    \[a_1+a_2+\dots+a_n.\]
\end{definition}
Sequences don't necessarily have any pattern, but later on in the section, we will explore special types of sequences, and special wayss to calculate their partial sums.
\section{Explicit vs. Recursive Definitions}
A definition for a sequence is a pattern for generating the $n$th term of the sequence. An explicit definition can directly calculate the $n$th term of a sequence. It contains any degree of $n$ in the definition, and only $n$. On the other hand, a recursive definition relies on past terms to calculate the $n$th term. Let's take a look at the sequence of all positive integers.
\begin{example}[Sequence of Positive Integers]
    The sequence of the first $n$ positive integers is
    \[a_n=\{1,2,\dots,n\}.\]
    The explicit definition of this sequence is
    \[a_n=n.\]
    The recursive definiton of this sequence is
    \[a_n=a_{n-1}+1\]
    \[a_1=1.\]
\end{example}
The most important thing to remember for recursive definitions is \textbf{ALWAYS DEFINE THE FIRST $k$ TERMS OF A RECURSIVE SEQUENCE}, where $k$ is the largest value of $a_{n-k}$ in the recursive definition. Otherwise, it's like trying to knock dominoes down: even if the dominoes are set up to known the subsequent one down, if you don't knock down the first one, the others will not fall down.
\begin{problem}
    Find the explicit and recursive definitions for the sequence 
    \[\{1, 2, 4, 8, 16, 32, 64,\dots\}\]
\end{problem}
\begin{problem}
    Find the recursive definition for the sequence
    \[\{-2, 3, 1, 4, 5, 9, 14,\dots\}\]
    As an optional challenge, try finding the explicit definition for this sequence.
\end{problem}
\section{Arithmetic Sequences}
The first special type of sequence is an arithmetic sequence, 
where every term is a constant amount away from the previous term. This brings us to the general definition of an arithmetic sequence.
\begin{definition}[The General Arithmetic Sequence]
    The general arithmetic sequence $A_n$ for two arbitrary constants $a, d$ is 
    \[A_n=\{a, a+d, a+2d, \dots, a+d(n-1)\}.\]
    We denote $d$ as the common difference.\\\\
    The explicit definition of that sequence is
    \[A_n=a+d(n-1).\]
    The recursive definition of that sequence is
    \[A_n=A_{n-1}+d\]
    \[A_n=a.\]
\end{definition}
The logical reasoning behind the explicit definition is that the $nth$ term of a sequence has $n-1$ differences before it, starting with $a$. The explanation for the recursive definition is also pretty intuitive; since the common difference is always constant, any given term in the sequence is $d$ more than the last \\\\
The common difference can easily be calculated by subtracting any two adjacent terms of an arithmetic sequence, as it is always constant.
\begin{example}[A Simple Arithmetic Sequence]
    A simple arithemtic sequence is 
    \[A_n=\{1, 3, 5, 7, 9, 11,\dots\}.\]
    The explicit defintion of that sequence is 
    \[A_n=1+2(n-1)=2n-1\]
    The recursive defintion of this sequence is
    \[A_n=A_{n-1}+2\]
    \[A_1=1.\]
\end{example}
Sometimes, you'll receive a tricky problem like this:
\begin{problem}
    Find the explicit definition for $A_n$, where
    \[A_3=5\]
    \[A_7=17.\]
\end{problem}
\begin{solution}
    We can use the general explicit defintion to solve this problem. Plugging in $A_3$ and $A_7$ gives us
    \[A_3=a+d(3-1)\]
    \[5=a+3d-3\]
    \[8=a+3d.\]
    \[A_7=a+d(7-1)\]
    \[17=a+7d-7\]
    \[24=a+7d.\]
    Subtracting our second equation from the first gives us
    \[(a+7d)-(a+3d)=24-8\]
    \[4d=16\]
    \[d=4.\]
    Plugging $d$ into the first equation gives us 
    \[8=a+3\cdot4\]
    \[a=-4.\]
    Therefore, the explicit defintion of $A_n$ is $-4+4(n-1)$.
\end{solution}
In general,
\[d=\frac{A_k-A_l}{k-l-1}\]
- formulas+derivations
- problem
\section{Geometric Sequences}
- explain geometric series
- formulas+derivations
- problem
\section{Sigma Notation}
- explain the importance of Sigma
- explain how to use sigma
- problem
\section{Infinite Series}
- explain the importance of Sigma
- explain divergence+convergence
- problem
\section{Interest}
- explain APR/Yield
- explain how sequences+series tie into it
- problem 

\begin{subappendices}
\section{Bonus: Sums of $n^k$ powers}
    - formulas, how to derive next
\end{subappendices}