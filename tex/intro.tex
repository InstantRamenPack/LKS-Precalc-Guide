\chapter{Introduction}

Welcome to the Honors Precalculus Survival Guide! This is a document containing a collection of concepts, formulas, and topics that will be covered in the M420 Honors Preacalculus course (at least in the 2023-24 curriculum). Each chapter, or "Unit," will run through all the ideas discussed, and provide a relatively brief and easy to understand explanation for some ideas, as well as calculations and derivations of formulas paired with written logic and reasoning. There will also be "bonus" chapters on subjects that won't really be tested on, but are cool and helpful to know.

I decided to write the Precalc Guide because there was a lot of demand for a guide for this class specifically, and because of the notorious difficulty for this class. I'll just say upfront: Honors Precalc isn't hard. That is, if you pay good attention and try to understand the "why" rather than the "what," everything falls into place. But I still wanted to make this to hopefully break the "harder" parts of this class into easier pieces and alleviate some stress behind this class.

In the end though, this was heavily inspired by Michael Y. '24, the author of the original Lakeside math course guide titled "Multivariable Calculus: A Survival Guide." Without his contributions to Lakeside math, this would not have been possible. I highly suggest checking his document out for all your Multi needs.

Because I am "by no means an expert at [pre]calculus," to quote Michael, there will inevitably be errors and confusing parts. If anything doesn't add up (get it?), feel free to let me know. 

In closing, I wish you all good luck on your journey through Precalculus. The document is titled such because it covers Honors Precalculus topics, but anyone is free to use this as a reference. However, Honors or not, Precalculus is not as terrifying as it seems. Like I always say, "As long as you study, you'll be fine." Prioritize sleep, pay attention, and enjoy the wild ride. It's a fun class that will lead you to life-changing discoveries as long as you just stay curious. 